\chapter{Natural Language Processing pipeline}\label{ch:NLP}
The Natural Language Processing (NLP) pipeline is the first subpipeline of the system. It is expected that natural language is given as input. Natural Language  Processing (NLP) is therefore used to process and analyze the input. This is done with the purpose of later transforming the input into executable actions. 
To give the reader a clearer understanding of the NLP tools used, an example of a tool is the POS tagger. Figure \ref{table:POS_TAG_example} shows how the POS-tagger is used to categorize each word into its grammatical category. 

\begin{table}[ht]
    \centering
    \includegraphics[width=13cm]{img/POs_TAG_example.png}
    \caption{Table illustrating the output of a POS tagger.}
    \label{table:POS_TAG_example}
\end{table}



\section{Analysis of natural language commands} \label{sec:app}

The input for the system pipeline is regarded as an instruction-based sentence. It is assumed that the basic parts of an instruction-based sentence are a verb and a noun. The verb is used to determine the action, and the noun is used to determine the subject of the action. The sentence "move the tool point" is an example of an instruction-based sentence. The verb is "go", and the noun is "robot". More nouns or numbers combined with units can be added to expand the instruction and further describe the action. The sentence "move the tool point to object A" is an example of an instruction sentence with one verb and two nouns. The verb is "go", the first noun is "tool point" and the second noun is "object A". This is only a simplified example of how instructions could be structured, since the complexity of language can be unlimitably complex by adding to instructions any number of verbs, nouns, prepositions, adjectives, or others. Natural language input can also consist of any number of these complex instructions.
For summarizations then there is two key informations. The first being, that instruction is likely to have a verb used to determine the action within the instruction. It is also likely to be accompanied by nouns specifying it.
The second piece of information is that sentence structure analysis is vital for the NLP pipeline, as the instructions can be arbitrarily complex.
The approach to analyzing the instruction-based natural language input is to use a sentence structure analyzer to find the important verbs, nouns, and others in the sentence. By stating all grammatical categories and having a structure analysis of the sentence, it is possible to determine the actions contained within the instructions. Using this information it is possible to figure out, what natural language processing tools is needed to solve the task. The next section introduces stanza, and the tools used in the NLP pipeline.

\section{Stanza Pipeline} \label{sec:Stanza}
The stanza pipeline is a Standford NLP package containing several different language models \cite{qi2020stanza}. The models make the stanza pipeline capable of several different NLP tasks. This makes the Stanza pipeline useful for using several NLP analysis tasks in one tool.

The following section is a short description of the tools selected as the NLP tools for the NLP pipeline used. A complete list of the given tool names is given below.

\begin{itemize}
    \item tokenize
    \item mwt
    \item pos
    \item depparse
\end{itemize}

The tools used are based on the neural network model named "en", which stands for English. This model was downloaded from the hugging face community website\footnote{https://huggingface.co/stanfordnlp/stanza-en}. For this thesis, it was important to find a model which made use of transformers, but the English, danish, and multilingual stanza language models did not use transformers. They were all based on LSTM models. Therefore the most popular model was chosen instead.


\subsection{Prerequisite NLP tasks (tokenize and mwt)} \label{sec:Prerequisite_NLP}
Most of the tools in the stanza pipeline require other basic NLP tools.
An essential tool for all other NLP tools is a tool for segmenting the input. The stanza pipeline uses two segmentation tools. Firstly, the tokenization module is used to segment the input into words. Secondly, the sentence segmentation module is used to segment the input into sentences. Another prerequisite task is "mwt". This is the multi-word token expansion module used to expand multi-word tokens into several single tokens. This is done by using a dictionary of multi-word tokens. An example could be the word "aren't" which can be broken into the two tokens "are not".
\subsection{Part of speech tagging (pos)} \label{sec:POS}
Stanzas Part of speech tagging module is a tool capable of categorizing words into their respective grammatical categories.
This is used to determine the meaning of each word by assigning it a grammatical category, like a noun or verb. This tool is essential for the parser in the next pipeline to extract the correct actions from the instruction. An example of POS tag processing is shown in table \ref{table:POS_TAG_example}.


\subsection{Dependency Parsing (depparse)} \label{sec:dep}
\begin{figure}[ht]
    \centering
    \includegraphics[width=13cm]{img/Dep_parser_structure.png}
    \caption{Figure illustrating an example of the dependency parser structure, where 'Move' and 'grab' are the roots of their respective sentences.}
    \label{fig:DEP_PARSER_exmaple}
\end{figure}


\begin{figure}[ht]
    \centering
    \includegraphics[width=13cm]{img/Dep_parser_example_let_go.png}
    \caption{Figure illustrating two verbs but one instruction}
    \label{fig:DEP_PARSER_exmaple_let_go}
\end{figure}
The dependency parsing tool creates connections between the words within the sentence and their respective syntactic heads. An understanding of a syntactic head is that traveling from word to syntactic head always leads to the root of the sentence. The root of the sentence is the most descriptive word in the sentence and does not have a syntactic head. If the input sentence is an instruction, then the root is almost always the instruction verb. This structure is shown in figure \ref{fig:DEP_PARSER_exmaple}. Figure \ref{fig:DEP_PARSER_exmaple} also shows that a relation is given between the two verbs. The dependency parser gives relations between all words and their syntactic heads, but only relations between verbs are used by the parser. The relations between verbs are used to determine if the sentence has more than one instruction. The relation between the two verbs shown in figure \ref{fig:DEP_PARSER_exmaple} is parataxis. Parataxis means that the two verbs describe two separate actions. This means that the two verbs are not dependent on each other and therefore should be treated as two separate instructions.
Figure \ref{fig:DEP_PARSER_exmaple_let_go} shows an example of a verb relation that indicates that 'go' is not an instruction by itself. This information is important, as 'go' should not be used by itself to indicate any actions.



\subsection{Stanza NLP output} \label{sec:stanzaOutput}
Using all of the given tools, the NLP tool is now capable of segmenting all of the words into their respective grammatical categories, along with passing structural information about the sentence. The output from the pos tagger, along with the output from the dependency parser, is the output of the NLP pipeline. The output is visualized as shown in figures \ref{fig:DEP_PARSER_exmaple} and \ref{fig:DEP_PARSER_exmaple_let_go}.
