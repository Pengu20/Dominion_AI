\chapter{Integration}\label{ch:int}

\begin{itemize}
    \item State that all the pieces for the project is now in place and that the final step is to integrate all the pieces together.
    \item introduce this chapter as the chapter where all the pieces are put together, and where the input and output of all the different pipelines are used.
    \item Show a diagram of the whole system and explain the the chapter will be based on this diagram
\end{itemize}

\section{System IO interface}\label{sec:Integration_IO}
\begin{itemize}
    \item Explain the terminal interface based on a image
    \item Explain what is expected as its input, and what the output could be.
    \item (its a simple terminal interface which can also create new windows where the demonstration happens)
\end{itemize}


\section{Natural langauge processing pipeline \- Stanza}\label{sec:Integration_stanza}
\begin{itemize}
    \item Show in a diagram where the natural langauge pipeline will be placed
    \item input: text
    \item output: text with NLP tags
\end{itemize}

\section{Parser}
\begin{itemize}
    \item introduce the section as having multiple purposes.\ therefore multiple subsections.
\end{itemize}

\subsection{NLP output deconstruction}\label{esc:Integration_decon}
\begin{itemize}
    \item Introduce this subsection as the place within the parser that handles the output from the NLP pipeline
    \item input: text with NLP tags
    \item output: verb sentences (list of verbs with all words connected to the verb as a full package)
\end{itemize}


\subsection{verb sentences to simple actions}\label{esc:Integration_parse_VS_SA}
\begin{itemize}
    \item This block is the place where the parser takes the verb sentences and converts them into simple actions by using dictionaries where the verbs are compared to the keys in the dictionary. \textcolor{red}{might want to add, that the verbs should be lemmatized before the key comparison}
    \item A description of the simple actions will be given, where it is explained that they are objects with specific parameters.
    \item input: verb sentences
    \item output: simple actions
\end{itemize}

\subsection{Simple actions to complex actions}\label{esc:Integration_parse_SA_CA}
\begin{itemize}
    \item This block is the place where the parser takes the simple actions and converts them into complex actions by merging the simple actions into a sequence.
    \item Explain the opportunity to merge newly formed complex actions with old complex actions made in prior
    \item input: simple actions
    \item output: Complex actions
\end{itemize}

\section{Kinematics for robot execution of actions}\label{sec:Integration_kin_for_robot}
\begin{itemize}
    \item Introduce this block as the place where the list of simple actions is used to generate the correct kinematical movements associated with the simple actions.
    \item Explain this part as a form of converter, as it takes the complex actions and converts them into a list of kinematical movements.
    \item Input: complex actions
    \item Output: kinematical movements
\end{itemize}


\section{Overview of full system design}\label{sec:Integration_overview}
\begin{itemize}
    \item Summarize the whole process as a pipeline which takes natural language text as its input, and outputs kinematics movements which is deployed of the UR5e.
    \item note the existence of the memory blocks which are used to store both information given as the natural language input and information which the system starts up with. Meaning that the natural language input is not only for manipulating the UR5e.\ and that there is opportunity to convey even more complex actions by using the memory blocks.\ as an example\ '\textit{execute the last given action without the need for the virtual demonstration beforehand}'
\end{itemize}
