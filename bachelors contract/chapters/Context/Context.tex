
\textbf{\Large{Context}}

A rise in the use of robots in the industry is currently taking place. Such 
movement towards automatic production is often referred to 
as the "fourth industrial revolution". More robots in the industry means 
more robot control. The fourth industrial revolution therefore puts 
pressure on the limits of human-robot collaboration (HRC)
\cite{4_0_Industrial_revolution}.
The expansion of robots into human workspaces has paved the way
for collaborative robots, which have the innate ability of HRC, as they are
significantly safer to work with than their fully 
automated past counterparts\cite{cobots}.
But even then, the use of HRC is not what it could be. As the 
level of collaboration within the industry described by Wilhelm Bauer, 
resides mostly at the coexistence and synchronization 
level\cite{HRC_levels}.
It's proposed that the lack of collaboration between the 
robot and the human stems from the complexity of work that a human
must perform to control the robot. Assuming the statement to be true,
then it's hypothesized that the use of natural language to control robots 
could enable, or at least ease, the transition into 
true collaboration between robots and humans in the industry. 



